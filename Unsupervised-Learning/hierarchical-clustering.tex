% --- 23-10-10 ---
\section{계층적 군집화}
계층적 군집화는 k-means 군집화와 다르게 군집의 개수 k를 설정해줄 필요가 없다. 대신 사용자가 두 그룹 내 관측치 사이의 쌍별 비유사도에 근거한, 관측치의 그룹 간 비유사도의 측정치를 지정하도록 요구한다. 두 그룹 혹은 두 집합들에 있는 관측치들 사이의 얼마나 멀고 가까운지에 대해 생각한다. HC(계층적 군집화)의 favorite한 method는 \textbf{ward method}이다. Ward의 최소 분산 방법은 Joe H. Ward, Jr.가 원래 제시한 목적 함수 접근 방식의 특별한 경우다. ward는 응집형(agglomerative) 계층적 군집화를 제안했다. 여기서 각 단계에서 병합할 클러스터 쌍을 선택하는 기준은 목적 함수의 최적 값을 기반으로 한다.  
% (https://en.wikipedia.org/wiki/Ward%27s_method 참고)

계층적 군집화는 2가지 전략을 갖고 있다. 응집형, 분열 두가지.
k-means 3일 때랑 4일 때의 차이? - k의 개수를 늘렸다고 해서 그 두 결과가 유기적인 관계를 갖지 않는다. 그러나 계층 군집에서는 그것이 성립한다. 따라서 partition들의 order를 생각해볼 수 있다.(수업자료 그림 참고) 그래서 좀 더 설명가능한 모델이 된다. 덴드로그램과 일반적인 tree의 차이는? - 이진트리의 경우 다음 세대로 갈 때 높낮이가 다를 필요가 없다. - 무슨 의미? - 이 높이가 두 분기점이 생기는 시점에서의 두개의 그룹의 비유사도가 프로포셔널하기 때문에, 즉 길이가 길면 차이가 크기 때문에 클러스터를 잘 구분해주기 때문이다.

덴드로그램은 결과가 단순히 나오는 것이 아니라 데이터의 전반적인 feature의 대한 아이디어를 얻을 수 있는 장점이 있다. 그럼에도 불구하고 일반적으로 한계점이 있긴하다. 왜? - 실제 데이터가 계층 구조가 있건 없건 알고리즘으로 뽑아내지만 계층적인정보를 무지성으로 만들어 주기 때문에 그 결과가 어떤 binary method를 쓰냐에 따라 결과가 달라진다. \textbf{\textit{cophenetic correlation coef}} 란? - 

% (내가 이해한대로 쭉 쓰길)

왜 N-1개만 distinct하니? - 다시 들어도 이해가 안감

식 (14.40)에 대한 고찰 - 다시 들어도 이해가 안감
